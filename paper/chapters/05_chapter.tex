\chapter{Experimente și rezultate}
\label{chap:experimente}

\textit{Acest capitol va fi completat cu rezultate cantitative după finalizarea antrenării PhysicsSSM.  Structura este pregătită.}

% ─────────────────────────────────────────────────────────────────────────
\section{Benchmark M1 --- Calitatea parsării}

TODO: rezultate din \texttt{scripts/benchmark\_m1\_quality.py} pe 200 de propoziții.

Metrici:
\begin{itemize}
    \item Entity extraction F1
    \item Action classification accuracy
    \item Spatial relation detection accuracy
\end{itemize}

% ─────────────────────────────────────────────────────────────────────────
\section{Benchmark M4 --- Calitatea mișcării}

TODO: FID, diversity, foot sliding rate.

Comparație:
\begin{itemize}
    \item MotionSSM fără PhysicsSSM (ablation)
    \item MotionSSM cu PhysicsSSM (modelul nostru)
    \item Raw KIT-ML retrieval (baseline)
\end{itemize}

% ─────────────────────────────────────────────────────────────────────────
\section{Benchmark M5 --- Aderența fizică}

TODO: ground penetration rate, contact consistency, self-collision events.

% ─────────────────────────────────────────────────────────────────────────
\section{Pipeline end-to-end}

TODO: 10 propoziții test, timp de procesare per propoziție, frame rate, calitate vizuală.

\begin{itemize}
    \item Showcase v4: 10 video-uri generate cu succes la 24 fps, durata 6 secunde fiecare
    \item Timp mediu per video: $\sim$45 secunde pe RTX 3050
\end{itemize}

% ─────────────────────────────────────────────────────────────────────────
\section{Analiza calitativă}

TODO: capturi de ecran din video-uri generate, comparație vizuală înainte/după PhysicsSSM.
