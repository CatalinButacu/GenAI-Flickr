\chapter{Concluzii}
\label{chap:concluzii}

\section{Contribuții}

\begin{enumerate}
    \item \textbf{PhysicsSSM} --- o arhitectură originală care amestecă modelarea temporală SSM cu constrângeri fizice printr-o poartă sigmoid învățată.  Antrenat pe KIT-ML cu pierdere de reconstrucție + penalizare de încălcări fizice.
    
    \item \textbf{Pipeline end-to-end text-to-physics-video} --- un sistem modular (M1--M8) care transformă o propoziție în video fizic verificat, fără intervenție umană.
    
    \item \textbf{Detecție de contact în buclă} --- integrarea interogărilor de contact PyBullet în bucla de simulare, cu clasificare sol/obiect și logare a forțelor.
    
    \item \textbf{Knowledge retrieval cu FAISS} --- îmbogățirea semantică a entităților cu proprietăți fizice din Visual Genome.
\end{enumerate}

\section{Limitări}

\begin{itemize}
    \item Humanoidul folosește Root State Injection (teleportare), nu forțe aplicate --- mișcarea nu este complet fizică
    \item Baza de cunoștințe conține doar 12 obiecte curate
    \item PhysicsSSM are 2.8M parametri --- un model mai mare ar putea fi mai expresiv, dar RTX 3050 limitează capacitatea
    \item Modulele M3 (Shap-E) și M8 (ControlNet) sunt opționale din cauza limitărilor VRAM
\end{itemize}

\section{Direcții viitoare}

\begin{itemize}
    \item Înlocuirea RSI cu forțe aplicate (physics-driven locomotion)
    \item Extinderea bazei de cunoștințe la 1000+ obiecte din Visual Genome complet
    \item Antrenarea pe Human3.6M și AMASS pentru diversitate mai mare
    \item Integrarea cu modele text-to-3D mai recente (InstantMesh, DreamGaussian)
    \item Studiul ablativ complet: PhysicsSSM vs. MotionSSM vs. difuzie
\end{itemize}
