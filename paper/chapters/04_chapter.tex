\chapter{Implementare}
\label{chap:implementare}

Acest capitol descrie codul, datele și procesul de antrenare.  Toate cifrele sunt reale --- extrase direct din loguri, nu estimate.

% ─────────────────────────────────────────────────────────────────────────
\section{Mediu de dezvoltare}

\begin{itemize}
    \item \textbf{Hardware}: NVIDIA RTX 3050 (4 GB VRAM), 16 GB RAM DDR4, Intel i5-12th gen
    \item \textbf{Software}: Python 3.12.8, PyTorch 2.5, CUDA 12.4, Windows 11
    \item \textbf{Dependențe cheie}: PyBullet 3.2.6, sentence-transformers, FAISS-cpu, imageio, scipy, transformers (HuggingFace)
    \item \textbf{Versionare}: Git, GitHub (\texttt{CatalinButacu/GenAI-Flickr})
\end{itemize}

% ─────────────────────────────────────────────────────────────────────────
\section{Structura codului}

Codebase-ul are \textbf{9.208 linii de Python} distribuite astfel:

\begin{table}[h]
\centering
\begin{tabular}{l r}
\hline
\textbf{Director} & \textbf{Linii} \\
\hline
M5 --- Physics Engine & 2.072 \\
M1 --- Scene Understanding & 1.730 \\
M4 --- Motion Generator & 1.638 \\
M4 --- SSM submodule & 580 \\
M8 --- AI Enhancer & 813 \\
pipeline.py (orchestrator) & 670 \\
M2 --- Scene Planner & 597 \\
shared/ (constante, vocabular) & 387 \\
M3 --- Asset Generator & 262 \\
data/ (data loading) & 227 \\
M7 --- Render Engine & 219 \\
\hline
\textbf{Total} & \textbf{9.208} \\
\hline
\end{tabular}
\caption{Distribuția liniilor de cod pe module}
\label{tab:loc}
\end{table}

Fiecare fișier Python are un header standard cu patru câmpuri:
\begin{itemize}
    \item \texttt{\#WHERE} --- cine îl apelează
    \item \texttt{\#WHAT} --- ce face
    \item \texttt{\#INPUT} --- ce primește
    \item \texttt{\#OUTPUT} --- ce produce
\end{itemize}

% ─────────────────────────────────────────────────────────────────────────
\section{Datele de antrenare}

\subsection{KIT Motion Language (KIT-ML)}

Datasetul principal pentru generarea de mișcare:
\begin{itemize}
    \item \textbf{4.888 secvențe} de motion capture la 20 fps
    \item \textbf{251 dimensiuni} per cadru (unghiuri articulare, poziție root, velocitate)
    \item Fiecare secvență are una sau mai multe descrieri text în engleză
    \item Split: 4.886 antrenare / 300 validare (2 secvențe cu lungime 0 excluse)
\end{itemize}

\subsection{Starea fizică derivată}

Pentru antrenarea PhysicsSSM, derivăm un vector de stare fizică de 64 dimensiuni din datele KIT-ML:

\begin{table}[h]
\centering
\begin{tabular}{c l}
\hline
\textbf{Canal} & \textbf{Semnificație} \\
\hline
0       & Gravitație ($-9{,}81$ m/s\textsuperscript{2}) \\
1       & Înălțimea pelvisului (m) \\
2       & Viteza verticală pelvis (m/s) \\
3--5    & Viteza root XYZ \\
6--8    & Accelerația root XYZ \\
9       & Energia cinetică normalizată \\
10--11  & Contactul picior stâng / drept (binar estimat) \\
12--23  & Zero-padding (rezervat) \\
24--63  & Zero-padding (extensii viitoare) \\
\hline
\end{tabular}
\caption{Formatul vectorului de stare fizică (64 dimensiuni)}
\label{tab:physics_state}
\end{table}

\subsection{Baza de cunoștințe (Knowledge Base)}

12 obiecte curate din Visual Genome, fiecare cu:
\begin{itemize}
    \item Nume canonic și aliasuri
    \item Dimensiuni tipice (H $\times$ W $\times$ L în metri)
    \item Masa (kg), materialul, coeficienți de frecare și restituție
    \item Prompt pentru generare mesh 3D
\end{itemize}

Index FAISS persistent (\texttt{IndexFlatIP}, cosine similarity) peste embedding-uri Sentence-BERT (384-dim).

% ─────────────────────────────────────────────────────────────────────────
\section{Antrenarea M1 --- T5 Parser}

\begin{itemize}
    \item Model: T5-small (60M parametri), pre-antrenat de Google
    \item Fine-tuning: 5 epoci, AdamW, learning rate $5 \times 10^{-5}$, amestec de triplete text$\rightarrow$JSON
    \item \textbf{Pierdere finală}: 0.062
    \item Checkpoint: \texttt{m1\_checkpoints/m1\_scene\_extractor\_v5}
    \item Timp de antrenare: $\sim$30 minute pe RTX 3050
\end{itemize}

% ─────────────────────────────────────────────────────────────────────────
\section{Antrenarea M4 --- MotionSSM}

\begin{itemize}
    \item Model: MotionSSM (4 straturi Mamba, $d_{\text{model}} = 256$, $d_{\text{state}} = 32$)
    \item Date: 4.886 secvențe KIT-ML
    \item 250 de epoci, batch size 16, AdamW, lr $5 \times 10^{-5}$
    \item \textbf{Pierderea cea mai bună}: 0.37 (validare)
    \item Checkpoint: \texttt{checkpoints/motion\_ssm/best\_model.pt}
\end{itemize}

% ─────────────────────────────────────────────────────────────────────────
\section{Antrenarea M4 --- PhysicsSSM (contribuția originală)}

\begin{itemize}
    \item Model: MotionSSM + PhysicsSSM + MotionProjector (2.789.275 parametri total)
    \item Funcția de pierdere: $\mathcal{L}_{\text{recon}} + 0{,}1 \cdot \mathcal{L}_{\text{fizică}}$
    \item Warmup: 500 de pași, OneCycleLR
    \item Early stopping cu patience = 15 epoci
\end{itemize}

Convergența observată (primele epoci):

\begin{table}[h]
\centering
\begin{tabular}{c c c c c}
\hline
\textbf{Epocă} & \textbf{Train Loss} & \textbf{Val Loss} & \textbf{Val Recon} & \textbf{Val Physics} \\
\hline
1  & 0.4593 & 0.2749 & 0.2539 & 0.2106 \\
2  & 0.1932 & 0.1217 & 0.1006 & 0.2113 \\
3  & 0.1229 & 0.0969 & 0.0757 & 0.2117 \\
\hline
\end{tabular}
\caption{Convergența PhysicsSSM (primele 3 epoci)}
\label{tab:physics_ssm_convergence}
\end{table}

\textit{Notă: antrenarea continuă. Tabelul va fi actualizat cu rezultatele finale.}
